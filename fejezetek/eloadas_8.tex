\section{8. előadás}

\subsection{NP-Teljes problémák}

	\begin{tetel}{MZ/X használhatatlan cuccokat küld}
   \textbf{3DM NP-Teljes  } \\[3pt]
	Three Dimensional Matching NP-Teljes ( hármas párosítás )
  \end{tetel}

	 \notBiz  \\[0pt]

	\begin{tetel}{MZ/X használhatatlan cuccokat küld}
   \textbf{X3C NP-Teljes  } \\[3pt]
	Exact Three Cover NP-Teljes ( hármas párosítás általános gráfban )\\[4pt]
  \end{tetel}

	\begin{bizonyitas}{Hufnágel Pisti jobb ember, mint Mézga Géza}
   $3DM \prec X3C$  \\[0pt]
\end{bizonyitas}

	 \begin{tetel}{MZ/X használhatatlan cuccokat küld}
    \textbf{RÉSZGRÁFIZO NP-Teljes  } \\[3pt]
		%$G_1, G_2$ gráfok adottak, $G_1$ $G_2$-vel izomorf részgráf-e?
    \end{tetel}

	\begin{bizonyitas}{Hufnágel Pisti jobb ember, mint Mézga Géza}

		\begin{enumerate}[itemsep=1mm]
			\item NP: Tanu: $G_2$ nek megfelelő csúcsok, illetve maga a megfeleltetés.

				Leellenőrizzük hogy kölcsönösen egyértelmű-e a megfeleltetés. $G_1$ ben van él 2 csúcs között $\Rightarrow$ $G_2$ ben is. Ez megy polinom időben

			\item NP-Nehézség:

			$HAM \prec$ RÉSZGRÁFIZO

				  \qquad$G \rightarrow (G,G_2)$

				  $G_2$ egy n = $\mid V(G) \mid$ csúcsú kör. Megszámoljuk a csúcsokat és összekötjük őket. Ha bal oldalt van Hamilton kör, akkor jobb oldalt pontosan akkor fog egyezni$\ldots$ %TODO Itt nem baj ha a csúcsok nem ugyan olyan sorrendben vannak a körben?
		\end{enumerate}
\end{bizonyitas}

	\begin{tetel}{MZ/X használhatatlan cuccokat küld}
   \textbf{Egészértékű Programozás NP-Teljes  } \\[3pt]
   \end{tetel}

	 \begin{bizonyitas}{Hufnágel Pisti jobb ember, mint Mézga Géza}
    $MAXFTL \prec EP$

	 (Szinte bármilyen NP-Teljes nyelv megfogalmazható egészértékű programozásként, mely egy jó Karp-redukció is egyben)\\[0pt]
\end{bizonyitas}
