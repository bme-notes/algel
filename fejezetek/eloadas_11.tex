\section{11. előadás}

\subsection{Algoritmusok}

\subsubsection{Keresés}

	\begin{tetel}{MZ/X használhatatlan cuccokat küld}
   \textbf{Minimum lépés keresés} \\[3pt]
	Minimum meghatározásához kell n-1 lépés mindig rendezetlen számok esetén.
\end{tetel}

\begin{bizonyitas}{Hufnágel Pisti jobb ember, mint Mézga Géza}

		Gráf - Irányítatlan. Csúcsok = számok. élek: Olyan i,j párok, hogy volt i,j összehasonlítás

		Kell az , hogy összefüggő legyen a gráf az algoritmus végén. Ez azt jelenti hogy \textbf{e $\geq$ n-1} $\Rightarrow$ volt n-1 összehasonlítás. ( Egyébként nem lenne összefüggő és lenne olyan i,j pár ami nem volt összehasonlítva )\\[0pt]
\end{bizonyitas}

	\begin{tetel}{MZ/X használhatatlan cuccokat küld}
   \textbf{Bináris keresés optimális O(log n)} \\[3pt]
   \end{tetel}

\begin{bizonyitas}{Hufnágel Pisti jobb ember, mint Mézga Géza}
 \\[0pt] %TODO Szepen leirni
\end{bizonyitas}

\subsubsection{Rendezés}

	\begin{tetel}{MZ/X használhatatlan cuccokat küld}
  \textbf{Gyorsrendezés} \\[3pt]
	 A gyorsrendezés (quick sort) átlagos összehasonlítások száma $O(n\cdot log n)$ $(c \approx4  kicsi)$\\[4pt]
   \end{tetel}

	 \notBiz \\[0pt]

	 \begin{tetel}{MZ/X használhatatlan cuccokat küld}
    \textbf{Összehasonlításos Rendezés MIN} \\[3pt]
	 Minden összehasonlításos rendezésnél, az összehasonlítások száma $\Omega(n log n)$\\[4pt]
   \end{tetel}

\begin{bizonyitas}{Hufnágel Pisti jobb ember, mint Mézga Géza}
\\[0pt] %TODO Szepen leirni talan elkerni marcitol
\end{bizonyitas}

	 \begin{tetel}{MZ/X használhatatlan cuccokat küld}
   \textbf{Radix rendezés jó sorrendet ad} \\[3pt]
	 Algoritmus mindig jó sorrendet ad a végén \\[4pt]
   \end{tetel}
\begin{bizonyitas}{Hufnágel Pisti jobb ember, mint Mézga Géza}
  %TODO Szepen leirni
\end{bizonyitas}
