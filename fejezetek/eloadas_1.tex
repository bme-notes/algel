\section{1. előadás}
\subsection{Becslések}

\begin{definicio}{Ordó}
  Legyen f,g $\mathbb{N} \rightarrow \mathbb{R}^+$ \\
  					 Az $f(n)$ függvény $O(g(n))$,ha van olyan $c > 0 \land n_0 > 0$ küszöb hogy: \\
  					 $f(n) \leq c \cdot g(n)\quad \forall n \geq n_0$ után\\[3pt]
  					 Jelben: $f(n) = O(g(n)) \Leftrightarrow f(n) \in O(g(n))$ \\[0pt]
\end{definicio}

\begin{definicio}{Omega}
  		Legyen f,g $\mathbb{N} \rightarrow \mathbb{R}^+$ \\
  					 Az $f(n)$ függvény $\Omega (g(n))$,ha van olyan $c > 0 \land n_0 > 0$ küszöb hogy: \\
  					 $f(n) \geq c \cdot g(n)\quad \forall n \geq n_0$ után.\\[6pt]
\end{definicio}

\begin{definicio}{Theta}
  Az $f(n)$ függvény $\Theta (g(n)), ha f(n) \in O(g(n)) \land \Omega (g(n)) is.$
\end{definicio}
