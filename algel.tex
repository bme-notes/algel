%%%% Beállítások, importok

\input{includes/header.tex}

\usepackage{relsize}
\usepackage{enumitem}

\newcommand{\Gorog}[1]{\mathlarger{\mathlarger{\mathlarger{#1}}}}
\newcommand{\notBiz}{\textbf{\textcolor{red}{$\neg$B }}}

%%%%
%%%%%%%
%%%% Ezeket változtasd meg!

\cim{Algoritmuselmélet}
\datum{2017. január 10.}
\szerzo{Kormány Zsolt}
\segitettek{Bognár Márton}

%%%%
%%%%%%%
%%%% Fedlap

\begin{document}
\begin{titlepage}
		\centering
		\vspace{5cm}\par
		\maketitle
		\large A jegyzet és annak forrása megtalálható a \texttt{bme-notes.github.io} weboldalon.
		\vfill
		Közreműködtek: \the\segitettek
		\normalsize
		% Bottom of the page
\end{titlepage}

%%%%
%%%%%%%
%%%% Tartalomjegyzés + előszó

\noindent \textbf{Kellemes vizsgázást!}

\tableofcontents{}

\section{Előszó}
Lórem ipszum

%%%%
%%%%%%%
%%%% Tényleges content
%%%% A fejezeteket a fejezetek/$NEV.tex elérési útvonalon keresi
%% Első "fejezet"
\ujfejezet{eloadas_1}
\ujfejezet{eloadas_2}
\ujfejezet{eloadas_3}
\ujfejezet{eloadas_4}
\ujfejezet{eloadas_5}
\ujfejezet{eloadas_6}
\ujfejezet{eloadas_7}
\ujfejezet{eloadas_8}
\ujfejezet{eloadas_9}
\ujfejezet{eloadas_10}
\ujfejezet{eloadas_11}
\ujfejezet{eloadas_12}
\ujfejezet{eloadas_13}

\end{document}
